
% License:
% CC BY-NC-SA 3.0 (http://creativecommons.org/licenses/by-nc-sa/3.0/)
%
%%%%%%%%%%%%%%%%%%%%%%%%%%%%%%%%%%%%%%%%%

%----------------------------------------------------------------------------------------
%	PACKAGES AND OTHER DOCUMENT CONFIGURATIONS
%----------------------------------------------------------------------------------------

\documentclass[paper=a4, fontsize=11pt]{scrartcl} % A4 paper and 11pt font size

\usepackage[T1]{fontenc} % Use 8-bit encoding that has 256 glyphs
\usepackage{fourier} % Use the Adobe Utopia font for the document - comment this line to return to the LaTeX default
\usepackage[english]{babel} % English language/hyphenation
\usepackage{amsmath,amsfonts,amsthm} % Math packages
\usepackage{lipsum} % Used for inserting dummy 'Lorem ipsum' text into the template

\usepackage{caption}
\usepackage{subcaption}
\usepackage{graphicx}

\usepackage{float}

\usepackage{blindtext} %for enumarations

\usepackage[]{hyperref}  %link collor

%talbe layout to the right
%\usepackage[labelfont=bf]{caption}
%\captionsetup[table]{labelsep=space,justification=raggedright,singlelinecheck=off}
%\captionsetup[figure]{labelsep=quad}

\usepackage{sectsty} % Allows customizing section commands
\allsectionsfont{\centering \normalfont\scshape} % Make all sections centered, the default font and small caps

\usepackage{fancyhdr} % Custom headers and footers
\usepackage{register} % Custom headers and footers
\pagestyle{fancyplain} % Makes all pages in the document conform to the custom headers and footers
\fancyhead{} % No page header - if you want one, create it in the same way as the footers below
\fancyfoot[L]{} % Empty left footer
\fancyfoot[C]{} % Empty center footer
\fancyfoot[R]{\thepage} % Page numbering for right footer
\renewcommand{\headrulewidth}{0pt} % Remove header underlines
\renewcommand{\footrulewidth}{0pt} % Remove footer underlines
\setlength{\headheight}{13.6pt} % Customize the height of the header

\numberwithin{equation}{section} % Number equations within sections (i.e. 1.1, 1.2, 2.1, 2.2 instead of 1, 2, 3, 4)
\numberwithin{figure}{section} % Number figures within sections (i.e. 1.1, 1.2, 2.1, 2.2 instead of 1, 2, 3, 4)
\numberwithin{table}{section} % Number tables within sections (i.e. 1.1, 1.2, 2.1, 2.2 instead of 1, 2, 3, 4)

%\setlength\parindent{0pt} % Removes all indentation from paragraphs - comment this line for an assignment with lots of text
\usepackage{enumitem}
\setenumerate[1]{label=\thesubsection.\arabic*.}
\setenumerate[2]{label*=\arabic*.}
\setlength\parskip{4pt}

%----------------------------------------------------------------------------------------
%	TITLE SECTION
%----------------------------------------------------------------------------------------

\newcommand{\horrule}[1]{\rule{\linewidth}{#1}} % Create horizontal rule command with 1 argument of height

\title{	
\normalfont \normalsize º
\horrule{0.5pt} \\[0.4cm] % Thin top horizontal rule
\huge  SafePMU Fault Tolerance Measures\\ % The assignment title
\horrule{2pt} \\[0.5cm] % Thick bottom horizontal rule
}

\author{ Guillem Cabo Pitarch} % Your name

\date{\today} % Today's date or a custom date

\usepackage{listings}
\lstdefinelanguage{VHDL}{
	morekeywords={
		library,use,all,entity,is,port,in,out,end,architecture,of,
		begin,and
	},
	morecomment=[l]--
}

\usepackage{textcomp}
\usepackage{xcolor}
\colorlet{keyword}{blue!100!black!80}
\colorlet{comment}{green!90!black!90}
\lstdefinestyle{vhdl}{
	language     = VHDL,
	basicstyle   = \ttfamily,
	keywordstyle = \color{keyword}\bfseries,
	commentstyle = \color{comment},
	framexleftmargin = 15pt
}

\usepackage{caption}
\DeclareCaptionFont{white}{\color{white}}
\DeclareCaptionFormat{listing}{%
	\parbox{\textwidth}{\colorbox{gray}{\parbox{\textwidth}{#1#2#3}}\vskip-4pt}}
\captionsetup[lstlisting]{format=listing,labelfont=white,textfont=white}
\lstset{frame=lrb,xleftmargin=\fboxsep,xrightmargin=-\fboxsep}


\begin{document}
%\nocite{*}
\maketitle % Print the title

\newpage
\tableofcontents

%----------------------------------------------------------------------------------------
%	Section 1
%----------------------------------------------------------------------------------------

\newpage
\section{General purpose of the module}

This module shall detect if an overflow of a given counter is about to happen, if any counter reaches the maximum value an interrupt shall be risen by the overflow module. Only one interrupt is required, the counter causing the interrupt shall be identified by a signal called overflow interrupt vector (over\_intr\_vect\_o) that will encode in one-hot fashion the offending counter or counters.\\
\\
Overflow detection shall be optional for each individual counter, hardware for overflow detection shall be present for all the counters but software shall be able to control through an overflow interrupt mask (over\_intr\_mask\_i) what counters can actually generate an interrupt.\\
\\
A dedicated enable signal shall be provided. Enable is active high.\\
\\
Interrupt can not be active if the unit is disabled. The unit is considered enabled if the unit is not in hard or soft reset and the enable is high.\\
\\
The whole unit shall use combinational logic, inputs and outputs are registered externally at the rising edge of the clock. The required inputs and outputs will be wires, names, types and default sizes are provided in section \ref{interface}.\\
\\

\section{Design placement}
\label{chapter2}
This modules is meant to be instantiated by the interface agnostic PMU (PMU\_raw.sv). Only one instance of this module is required. The application-specific interface shall take care of the  write enable of this module.
%This module is meant to be a blackbox inside the chisel code. It belongs to the Drac class in the \emph{rocket.scala} file. We can have as many instances of this module as cores are instantiated in the SoC. Currently only single core operation has been tested. 





\section{Parameters}
\label{chapter3}
This unit uses several parameters. \textbf{DATA\_WIDTH} defines the size of the data registers.\\
\textbf{WEIGHTS\_WIDTH} defines the size of the registers that store the estimated contention that a given event causes.\textbf{ N\_CORES} is used to configured to replicate the internal logic of the RDC to support several cores. \textbf{CORE\_EVENTS} sets the number of events that are track for each core.\\
\\
Internal parameters are used to perform padding between signals of different width when need to be used together. Such local parameter are\textbf{ O\_D\_0PAD, D\_W\_0PAD, O\_W\_0PAD }further details are provided in the source code comments. \\
\\
Given the previous parameters the unit shall generate correct RTL without any manual modification to the source code.\\
\\
\section{Configuration options}
\label{chapter 4}
Table \ref{generics} shows the configuration parameters exposed by \textit{ahb\_wrapper.vhd}. Given the current development status, changes to the configuration options may require manual modifications to internal modules and software drivers. Future releases will expose parameters to enable individual SafeSU features and allow for more flexibility.\\
\\
\begin{table}[H]
	\caption{Configuration options (VHDL ports)}
	\label{generics}
	\centering
	\begin{small}
		\begin{tabular}{|l|l|l|l|}
			\hline
			\textbf{Generic} & \textbf{Function}  & \textbf{Allowed range}  & \textbf{Default}\\
			\hline
			HADDR & AHB base address &0 to 16\#fff\# & 0\\
			\hline
			HMASK & AHB address mask &0 to 16\#fff\# & 16\#fff\#\\
			\hline
			N\_REGS & Total of accessible registers & 2 to 64 & 43\\
			\hline
			SafeSU\_COUNTERS & Number of generic counters. Same as crossbar outputs& 1 to 32 &24 \\
			\hline
			N\_SOC\_EV &SoC signals. Inputs to the crossbar& 1 to 64 & 32 \\
			\hline
			REG\_WIDTH &Size of registers and counters & 32 \textbf{or} 64 & 32 \\
			\hline
		
		\end{tabular}
	\end{small}
\end{table}

\section{Reset behavior}

The module contains two reset signals. One synchronous active high reset enabled by software called softrst\_i, the other is the global reset, called rstn\_i it is asynchronous and active low.\\
\\
Any reset when active shall clear any internal register and set them to 0. Interruptions shall become inactive while the unit is in reset.





%----------------------------------------------------------------------------------------
%	Section 2
%----------------------------------------------------------------------------------------






\end{document}
